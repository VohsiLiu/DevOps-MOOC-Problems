\subsubsection*{\S 软件架构演化}
\setcounter{problemname}{0}

\begin{problem}
	下面关于软件架构的描述哪个是不正确的?
	\uline{B}    
    % \vspace{-0.8em}
    % \begin{multicols}{2}
        \begin{enumerate}[label=\Alph*.]
            \item 软件架构包含一系列重要决策,包括软件组织、构成系统的结构要素等。
            \item 软件架构是一组特定的架构元素,包括处理元素、数据元素和上下文元素。
            \item 软件架构包括系统组件、连接件和约束的集合。
            \item 软件架构即一系列重要的设计决策。
        \end{enumerate}
    % \end{multicols}
    % \vspace{-1em}
\end{problem}



\begin{problem}
	在应用分层架构的软件系统中,最先处理外部请求的是:
	\uline{B}    
    \vspace{-0.8em}
    \begin{multicols}{4}
        \begin{enumerate}[label=\Alph*.]
            \item 数据层
            \item 表现层
            \item 应用层
            \item 业务层
        \end{enumerate}
    \end{multicols}
    \vspace{-1em}
\end{problem}



\begin{problem}
	以下哪个关于面向服务架构的描述是错误的?
	\uline{B}    
    % \vspace{-0.8em}
    % \begin{multicols}{2}
        \begin{enumerate}[label=\Alph*.]
            \item 面向服务架构包含服务提供者组件和服务消费者组件
            \item 面向服务架构是一个集中式组件的集合
            \item 在SOA中,服务消费者消费其他组件提供的服务不需要知道其具体的实现细节
            \item SOA依赖企业服务总线为服务间的相互调用提供支持环境
        \end{enumerate}
    % \end{multicols}
    % \vspace{-1em}
\end{problem}



\begin{problem}
	‌以下对于微服务优点的描述中,哪一个是错误的?
	\uline{C}    
    \vspace{-0.8em}
    \begin{multicols}{2}
        \begin{enumerate}[label=\Alph*.]
            \item 不同的微服务可以使用不同的语言进行开发
            \item 微服务可以使用RPC进行服务间通信
            \item 微服务系统测试变得非常简单
            \item 单个微服务很简单,只关注一个业务功能
        \end{enumerate}
    \end{multicols}
    \vspace{-1em}
\end{problem}



\begin{problem}
	在微服务架构中,ZooKeeper的主要作用是?
	\uline{B}    
    \vspace{-0.8em}
    \begin{multicols}{4}
        \begin{enumerate}[label=\Alph*.]
            \item 开发服务
            \item 注册服务
            \item 封装服务
            \item 调用服务
        \end{enumerate}
    \end{multicols}
    \vspace{-1em}
\end{problem}



\begin{problem}
	除Spring Boot之外,主流的微服务开发框架还有什么?
	\uline{A}    
    \vspace{-0.8em}
    \begin{multicols}{4}
        \begin{enumerate}[label=\Alph*.]
            \item Apache Dubbo
            \item Django
            \item MyBaits
            \item Kubernetes
        \end{enumerate}
    \end{multicols}
    \vspace{-1em}
\end{problem}



\begin{problem}
	在组成派看来,软件架构是指?
	\uline{ACD}    
    % \vspace{-0.8em}
    % \begin{multicols}{2}
        \begin{enumerate}[label=\Alph*.]
            \item 软件架构包括系统组件、连接件和约束的集合。
            \item 软件架构是一系列重要决策的集合,包括构成系统的结构要素及其接口的选择。
            \item 软件架构由软件元素、这些元素的外部可见属性,以及元素之间的关系组成。
            \item 软件架构将系统定义为计算组件及组件间的交互。
        \end{enumerate}
    % \end{multicols}
    % \vspace{-1em}
\end{problem}



\begin{problem}
	分层架构将软件系统的组件分成多个互不重叠的层,包括:
	\uline{BC}    
    \vspace{-0.8em}
    \begin{multicols}{4}
        \begin{enumerate}[label=\Alph*.]
            \item 物理层
            \item 持久层
            \item 数据层
            \item 应用层
        \end{enumerate}
    \end{multicols}
    \vspace{-1em}
\end{problem}



\begin{problem}
	分层架构模式的缺点包括:
	\uline{ABCD}    
    \vspace{-0.8em}
    \begin{multicols}{2}
        \begin{enumerate}[label=\Alph*.]
            \item 由于层间依赖关系,软件系统的可扩展性差
            \item 代码调整通常比较麻烦
            \item 软件升级需要暂停整个服务
            \item 不易于持续发布和部署
        \end{enumerate}
    \end{multicols}
    \vspace{-1em}
\end{problem}



\begin{problem}
	以下哪几个不是面向服务架构强调的实现原则?
	\uline{BC}    
    \vspace{-0.8em}
    \begin{multicols}{4}
        \begin{enumerate}[label=\Alph*.]
            \item 服务解耦
            \item 服务去中心化
            \item 服务简单
            \item 服务自治
        \end{enumerate}
    \end{multicols}
    \vspace{-1em}
\end{problem}



\begin{problem}
	‌以下选项中,哪些属于微服务架构的特点?
	\uline{ABCD}    
    \vspace{-0.8em}
    \begin{multicols}{4}
        \begin{enumerate}[label=\Alph*.]
            \item 围绕业务能力组织
            \item 基础设施自动化
            \item 通过服务组件化
            \item 内聚和解耦
        \end{enumerate}
    \end{multicols}
    \vspace{-1em}
\end{problem}



\begin{problem}
	以下选项中,API网关模式的优点有哪些?
	\uline{ABC}    
    % \vspace{-0.8em}
    % \begin{multicols}{4}
        \begin{enumerate}[label=\Alph*.]
            \item 将从客户端调用多项服务的逻辑转换为从API网关处调用,以简化整个客户端
            \item 为每套客户端提供最优API
            \item 确保客户端不受服务实例位置的影响
            \item 增加请求往返次数
        \end{enumerate}
    % \end{multicols}
    % \vspace{-1em}
\end{problem}



\begin{problem}
	与面向服务架构相关的Web服务标准包括:
	\uline{ABC}    
    \vspace{-0.8em}
    \begin{multicols}{4}
        \begin{enumerate}[label=\Alph*.]
            \item SOAP
            \item UDDI
            \item WSDL
            \item UML
        \end{enumerate}
    \end{multicols}
    \vspace{-1em}
\end{problem}



\begin{problem}
	单体应用的所有功能都被集成在一起作为一个单一的单元。
    \hfill (\ding{51})
\end{problem}



\begin{problem}
	单体架构更多地作为应用的部署架构,单体应用只运行在一个进程中。
    \hfill (\ding{55})
\end{problem}



\begin{problem}
	微服务架构架构风格是一种将一个单一应用程序开发为一个小型服务的方法。
    \hfill (\ding{55})
\end{problem}



\begin{problem}
	本质上,微服务架构是SOA的一种扩展。
    \hfill (\ding{51})
\end{problem}



\begin{problem}
	核心模式即针对采用微服务系统在通用场景下的所有问题,所使用的成熟的架构解决方案集合。
    \hfill (\ding{55})
\end{problem}

\subsubsection*{\S 云原生与容器技术}
\setcounter{problemname}{0}

\begin{problem}
	下列哪项不是Docker容器的特点:
	\uline{C}    
    \vspace{-0.8em}
    \begin{multicols}{2}
        \begin{enumerate}[label=\Alph*.]
            \item 创建速度很快
            \item 可以共享操作系统的资源
            \item 启动时间是分钟级
            \item 资源使用较少
        \end{enumerate}
    \end{multicols}
    \vspace{-1em}
\end{problem}



\begin{problem}
	下列哪项不是Docker的网络模式
	\uline{B}    
    \vspace{-0.8em}
    \begin{multicols}{4}
        \begin{enumerate}[label=\Alph*.]
            \item Bridge 模式
            \item 其他全是
            \item None模式
            \item Host模式
        \end{enumerate}
    \end{multicols}
    \vspace{-1em}
\end{problem}



\begin{problem}
    以下哪些是Docker的存储驱动:
	\uline{A}    
    \vspace{-0.8em}
    \begin{multicols}{4}
        \begin{enumerate}[label=\Alph*.]
            \item 其他都是
            \item verlayFS
            \item Device mapper
            \item AUFS
        \end{enumerate}
    \end{multicols}
    \vspace{-1em}
\end{problem}



\begin{problem}
	以下哪个命令可以查看当前运行容器:
	\uline{B}    
    \vspace{-0.8em}
    \begin{multicols}{4}
        \begin{enumerate}[label=\Alph*.]
            \item docker top
            \item docker ps
            \item docker run
            \item docker logs
        \end{enumerate}
    \end{multicols}
    \vspace{-1em}
\end{problem}



\begin{problem}
	Kubernetes集群将元数据保存在以下哪个组件:
	\uline{B}    
    \vspace{-0.8em}
    \begin{multicols}{4}
        \begin{enumerate}[label=\Alph*.]
            \item Kubelet
            \item Etcd
            \item 以上都不是
            \item Kube-apiserver
        \end{enumerate}
    \end{multicols}
    \vspace{-1em}
\end{problem}



\begin{problem}
	以下哪些是Kubernetes的控制器:
	\uline{C}    
    \vspace{-0.8em}
    \begin{multicols}{2}
        \begin{enumerate}[label=\Alph*.]
            \item ReplicaSet
            \item Deployment
            \item Both ReplicaSet and Deployment
            \item Rolling Updates
        \end{enumerate}
    \end{multicols}
    \vspace{-1em}
\end{problem}



\begin{problem}
	以下哪些是Kubernetes的核心概念
	\uline{A}    
    \vspace{-0.8em}
    \begin{multicols}{4}
        \begin{enumerate}[label=\Alph*.]
            \item 其他都是
            \item Services
            \item Volumes
            \item Pods
        \end{enumerate}
    \end{multicols}
    \vspace{-1em}
\end{problem}



\begin{problem}
    Kubernetes里面的Replication控制器的职责是:
	\uline{D}    
    \vspace{-0.8em}
    \begin{multicols}{2}
        \begin{enumerate}[label=\Alph*.]
            \item 当已存在的Pod异常退出后,创建新的Pod
            \item 帮助达到预期的状态
            \item 删除或者更新多个Pod
            \item 其他都是
        \end{enumerate}
    \end{multicols}
    \vspace{-1em}
\end{problem}



\begin{problem}
	如何通过命令行创建一个容器?
	\uline{C}    
    \vspace{-0.8em}
    \begin{multicols}{4}
        \begin{enumerate}[label=\Alph*.]
            \item docker create
            \item docker start
            \item docker run
            \item docker poll
        \end{enumerate}
    \end{multicols}
    \vspace{-1em}
\end{problem}



\begin{problem}
	Dockerfile中的命令 RUN, CMD 和ENTRYPOINT几者有何区别?
	\uline{B}    
    % \vspace{-0.8em}
    % \begin{multicols}{4}
        \begin{enumerate}[label=\Alph*.]
            \item ENTRYPOINT 配置容器启动时运行的命令
            \item 其他都是
            \item RUN 执行命令并创建新的镜像层,RUN 经常用于安装软件包。
            \item CMD 设置容器启动后默认执行的命令及其参数,但 CMD 能够被 docker run 后面跟的命令行参数替换
        \end{enumerate}
    % \end{multicols}
    % \vspace{-1em}
\end{problem}



\begin{problem}
	使用Kubernetes带来的好处有哪些?
	\uline{B}    
    \vspace{-0.8em}
    \begin{multicols}{4}
        \begin{enumerate}[label=\Alph*.]
            \item 横向扩展
            \item 其他都是
            \item 自动调度
            \item 自动回滚
        \end{enumerate}
    \end{multicols}
    \vspace{-1em}
\end{problem}



\begin{problem}
	以下哪项用于确保pod不会被调度到不适当的节点上?
	\uline{A}    
    \vspace{-0.8em}
    \begin{multicols}{2}
        \begin{enumerate}[label=\Alph*.]
            \item Taints 和 Tolerations
            \item 以上都不是
            \item TTaints
            \item Tolerations
        \end{enumerate}
    \end{multicols}
    \vspace{-1em}
\end{problem}



\begin{problem}
	Docker容器的状态有:
	\uline{BD}    
    \vspace{-0.8em}
    \begin{multicols}{4}
        \begin{enumerate}[label=\Alph*.]
            \item Paused
            \item Running
            \item Restarting
            \item Exited
        \end{enumerate}
    \end{multicols}
    \vspace{-1em}
\end{problem}



\begin{problem}
	关于Kubernetes的namespace:命名空间是在多个用户之间划分群集资源的方法
    \hfill (\ding{51})
\end{problem}



\begin{problem}
    以下描述是否正确:多步构建允许在Dockerfile中使用多个FROM指令。两个FROM指令之间的所有指令会生产一个中间镜像,最后一个FROM指令之后的指令将生成最终镜像。中间镜像中的文件可以通过\verb|COPY --from=<image-number>|指令拷贝,其中\verb|image-number|为镜像编号,0为第一个基础镜像。没有被拷贝的文件都不会存在于最终生成的镜像,这样可以减小镜像大小,同时避免出现安全问题。
    \hfill (\ding{51})
\end{problem}




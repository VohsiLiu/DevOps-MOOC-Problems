\clearpage
\subsubsection*{\S\ 简答题}
\setcounter{problemname}{0}

\renewenvironment{problem}{\stepcounter{problemname}\par\noindent\textbf{\arabic{problemname}.\,}}{}

\begin{problem}
请描述一个简单的持续交付流水线所包含的基本步骤。
\end{problem}

\begin{solution}
开发 $\rightarrow$ 版本控制 $\rightarrow$ 代码检查 $\rightarrow$ 构建 $\rightarrow$ 自动化测试 $\rightarrow$ 打包 $\rightarrow$ 远程仓库发布 $\rightarrow$ 自动化部署
\end{solution}




\begin{problem}
请列举出敏捷软件开发中常见的3种开发方法。
\end{problem}

\begin{solution}
\begin{enumerate}[label=\arabic*.]
    \item 极限编程(简称XP),是一种近螺旋式的开发方法,它将复杂的开发过程分解为一个个相对比较简单的小周期;通过积极的交流、反馈以及其它一系列的方法,开发人员和客户可以非常清楚开发进度、变化、待解决的问题和潜在的困难等,并根据实际情况及时地调整开发过程。
    \item 精益开发,思想起源于丰田公司,旨在创造价值的目标下,通过改良流程不断地消除浪费。这种方法现已被广泛用于生产制造管理,对于IT系统建设,精益开发的常用工具模型是价值流模型。
    \item Scrum 是一个用于开发和维护复杂产品的框架,是一个增量的、迭代的开发过程。Scrum以经验性过程控制理论(经验主义)做为理论基础的过程。经验主义主张知识源于经验,以及基于已知的东西做决定。Scrum 采用迭代、增量的方法来优化可预见性并控制风险。
\end{enumerate}
\end{solution}



\begin{problem}
请描述敏捷软件开发宣言内容?
\end{problem}

\begin{solution}
    \vspace{-0.8em}
    \begin{multicols}{2}
        \begin{itemize}
            \item 个体和互动 胜过 流程和工具
            \item 可以工作的软件 胜过 详尽的文档
            \item 客户合作 胜过 合同谈判
            \item 响应变化 胜过 遵循计划
        \end{itemize}
    \end{multicols}
    \vspace{-1em}
    \vspace{-0.4em}
    \begin{itemize}
        \item 也就是说,尽管右项有其价值,我们更重视左项的价值
    \end{itemize}
\end{solution}



\begin{problem}
请解释一下什么是Kanban方法中的WIP?为什么要限制WIP?
\end{problem}

\begin{solution}
WIP全称是work in progress,表示在途工作量,即同时进行中的工作数量。

限制WIP可以让团队成员更佳专注与手边的工作,减少工作切换所造成的浪费,因而可以加速工作完成的时间。此外,当工作卡住而无法完成的时候,开发团队不能以此为借口忽略这些被卡住的工作(因为有WIP限制,不能无限制的一直拿工作而不完成它),此时就有可能会有人闲置下来,而影响工作流程与产能。鼓励团队成员一起解决问题,排除阻碍。可以更均衡的工作产出,避免前期开发后期集中或者压缩测试时间,能够更稳健的长久的进行敏捷实践。
\end{solution}



\begin{problem}
什么叫做面向用户的质量观?这种观点对软件开发有什么影响?
\end{problem}

\begin{solution}
 面向用户的质量观是定义质量为满足用户需求的程度。这个定义中需要进一步明确:用户究竟是谁?用户需求的优先级是什么?这种用户的优先级对软件产品的开发过程产生什么样的影响?怎样来度量这种质量观下的质量水平?

在软件开发中,用缺陷管理来替代质量管理,高质量产品也就意味着要求组成软件产品的各个组件基本无缺陷。缺陷消除的平均代价随着开发过程的进展会显著增加,各个组件的高质量是通过高质量评审来实现的。
\end{solution}



\begin{problem}
DevOps有哪些常见的质量手段有助于确保最终软件服务的质量?
\end{problem}

\begin{solution}
\begin{enumerate}[label=\arabic*.]
    \item 持续集成。持续集成 (CI) 是一个开发过程,每天多次将代码集成到共享存储库中。借助自动化测试,CI 帮助允许团队及早识别错误、轻松定位问题,提高了软件质量并缩短了交付时间。
    \item 持续部署。通过评估拉取请求并将它们组合到主分支,持续部署为开发人员提供了对流水线末端产品的的关注。它允许企业快速部署、验证新功能,并在测试自动化完成后立即做出响应。有了持续部署流水线,一旦客户提交质量问题,团队就可以轻松处理新版本的错误,因为每个版本都是小批量交付的。
    \item 持续测试。在 CI/CD 工作流中,构建往往以小批量进行。因此,为每次构建,手动运行测试用例会非常耗时。持续测试借助自动化手段,尽早、逐步和充分地执行测试,发现问题解决问题。
    \item 自动化。借助强大的部署自动化手段和标准化的环境管理来降低部署操作的成本,确保部署任务的可重复性,减少部署出错的可能性。
    \item DevOps 致力于在整个开发过程中的每一个环节都引入QA 和测试管理,使它们成为质量的推动者,并确保产品符合利益相关者和用户所设定的质量标准。QA 实际上被认为是DevOps 中非常关键的组件,甚至于DevOps 强调质量保证是每个人的责任。
\end{enumerate}
\end{solution}



\begin{problem}
请谈谈微服务架构与面向服务的架构存在哪些异同?
\end{problem}

\begin{solution}
\begin{enumerate}[label=\arabic*.]
    \item 微服务是一种软件的架构风格,面向服务的架构(SOA)不是一种特定的技术,而是一种分布式计算的软件设计方法。
    \item 微服务简单的说就是组合化,它的每部分需要实现的功能可以有不同的小程序单独构成,然后相互之间协同实现一个大的目标。这个角度上来说,两者是一脉相承的,但是面向服务的架构,没有微服务的分离度高,相互之间的关联度还是相对较高。
    \item 微服务相比较来说,在各个组件上可以使用不一样的编程语言。
    \item 微服务更加关注于解耦,不追求系统之间的相关性。
    \item 微服务的系统发生改变只需要构建一个新的服务,简单快捷,但是SOA则需要对整个系统进行修改。
    \item 微服务使用的协议一般都是轻量级的协议,就像HTTP、Thrift API等协议,但是SOA则是使用更为复杂多样的多种消息协议。微服务的容错性能会更好,即使一个微服务出现问题,其他的微服务也会正常工作。
    \item 每个微服务都有单独的数据库,SOA则是共享一个数据库。
    \item 微服务的规模更小,SOA则是一个较大的规模。SOA可以是一个整体,也可以是多个微服务组成的。
\end{enumerate}
\end{solution}



\begin{problem}
从你的理解出发,谈谈为什么微服务架构具有高可用性、灵活性等优点?
\end{problem}

\begin{solution}
微服务架构普遍被采用于云原生应用、无服务器计算、以及使用轻量级容器部署的应用等,根据Fowler的观点,由于服务数量众多(与单体应用实现相比),为了有效地开发、维护和运营这类应用,去中心化的持续交付和带有整体服务监控的DevOps是必要的。 遵循这种方法的一个合理性结果是,单独的微服务可以单独扩展。在单体应用架构方法中,一个支持三个功能的应用,即使只有其中一个功能需要添加资源约束,也需要对其进行整体的扩展。微服务则不同,只需要对有资源约束需求的微服务进行扩展, 这样就带来了资源和成本的优化。

微服务并不是单体应用中的一个层,相反,它是一个自成一体的业务功能,具有明确的接口,可以通过自己的内部组件实现分层架构。从策略的角度来看,微服务架构本质上遵循了Unix的“做一件事,做好一件事 ”的理念,改变应用程序的一小部分只需要重建和重新部署一个或少量的服务即可,å使用可直达独立部署服务的精细化接口、业务驱动开发(如领域驱动设计)等原则。
\end{solution}
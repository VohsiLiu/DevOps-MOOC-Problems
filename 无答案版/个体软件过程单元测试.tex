\subsubsection*{\S 个体软件过程单元测试}
\setcounter{problemname}{0}

\begin{problem}
    下述各个度量项中,哪一个不是PSP的基本度量项?
    %\uline{B}    
    \vspace{-0.8em}
    \begin{multicols}{4}
        \begin{enumerate}[label=\Alph*.]
            \item 缺陷
            \item 风险
            \item 规模
            \item 时间
        \end{enumerate}
    \end{multicols}
    \vspace{-1em}
\end{problem}



\begin{problem}
    关于面向用户的质量观,我们应该关注如下哪些问题:
    %\uline{ACD}    
    % \vspace{-0.8em}
    % \begin{multicols}{2}
        \begin{enumerate}[label=\Alph*.]
            \item 用户期望是否有优先级?
            \item 界面和可操作性是首要的,因为这是用户能直接感受到的。
            \item 真实用户是谁?
            \item 用户期望的优先级对软件开发的影响?
        \end{enumerate}
    % \end{multicols}
    % \vspace{-1em}
\end{problem}



\begin{problem}
	PSP当中为什么用缺陷管理替代质量管理?下述说法中正确的是:
	%\uline{BD}    
    % \vspace{-0.8em}
    % \begin{multicols}{4}
        \begin{enumerate}[label=\Alph*.]
            \item 因为缺陷管理和质量管理其实是一回事。
            \item 因为单纯质量管理很难操作。
            \item 因为缺陷管理相关的活动(例如,测试等)本来就是软件开发中必须要开展的活动。
            \item 因为缺陷往往对应了面向用户质量观中的首要用户期望。
        \end{enumerate}
    % \end{multicols}
    % \vspace{-1em}
\end{problem}



\begin{problem}
	关于PROBE估算法,下述各种说法中,不正确的有哪些?
	%\uline{ABD}    
    % \vspace{-0.8em}
    % \begin{multicols}{4}
        \begin{enumerate}[label=\Alph*.]
            \item PROBE估算结果带着小数,肯定不准确,因而,不应该在项目估算的时候使用。
            \item PROBE不能给出精确估算,因而适合用来跟用户讨论需求和规模。
            \item PROBE方法不能用来估算质量。
            \item PROBE方法不需要历史数据。
        \end{enumerate}
    % \end{multicols}
    % \vspace{-1em}
\end{problem}



\begin{problem}
	关于质量路径(Quality Journey),下列说法中哪些不恰当。
	%\uline{AD}    
    % \vspace{-0.8em}
    % \begin{multicols}{4}
        \begin{enumerate}[label=\Alph*.]
            \item 质量路径中所列举的方法都是提升开发质量的有效手段,可以随意选择使用。
            \item 高质量软件产品最终还是需要依赖测试来确保。
            \item 进入测试之前的高质量,是获得测试之后高质量软件系统的前提条件。
            \item 质量路径与个体软件工程师无关,是团队层面的集体努力。
        \end{enumerate}
    % \end{multicols}
    % \vspace{-1em}
\end{problem}



\begin{problem}
	关于评审检查表,下述说法中不恰当的是:
	%\uline{CD}    
    % \vspace{-0.8em}
    % \begin{multicols}{2}
        \begin{enumerate}[label=\Alph*.]
            \item 评审检查表应该是个性化的
            \item 评审检查表应该定期更新
            \item 项目团队所有人应该共用一份评审检查表,体现统一性
            \item 评审检查表应该保持稳定,确保缺陷不会被遗漏
        \end{enumerate}
    % \end{multicols}
    % \vspace{-1em}
\end{problem}



\begin{problem}
    关于PQI,下述说法中不恰当的是:
	%\uline{BD}    
    % \vspace{-0.8em}
    % \begin{multicols}{4}
        \begin{enumerate}[label=\Alph*.]
            \item PQI可以用来辅助判断模块开发的质量
            \item PQI五个分指标都可以超过1.0,比如,设计时间多于编码时间的时候,该分指标就超过1.0了
            \item PQI可以为过程改进提供依据
            \item PQI越高越好,最好达到1.0
        \end{enumerate}
    % \end{multicols}
    % \vspace{-1em}
\end{problem}



\begin{problem}
	关于评审,下述说法中不恰当是:
	%\uline{CD}    
    % \vspace{-0.8em}
    % \begin{multicols}{4}
        \begin{enumerate}[label=\Alph*.]
            \item 代码的个人评审也应该通过评审检查表来进行。
            \item 如果安排了代码的小组评审,那么代码个人评审就可以不用做。
            \item 代码的个人评审最好交叉进行,因为阅读自己代码容易产生思维定式,不利于缺陷发现。
            \item 代码的个人评审应该安排在单元测试之后,确保评审对象有着较高的质量,提升评审价值。
        \end{enumerate}
    % \end{multicols}
    % \vspace{-1em}
\end{problem}



\begin{problem}
	关于质量的各种定义当中,下述哪些质量属性属于内部属性?
	%\uline{AD}    
    \vspace{-0.8em}
    \begin{multicols}{4}
        \begin{enumerate}[label=\Alph*.]
            \item 可扩展性
            \item 安全性
            \item 可靠性
            \item 可移植性
        \end{enumerate}
    \end{multicols}
    \vspace{-1em}
\end{problem}



\begin{problem}
	PSP鼓励使用瀑布型生命周期模型。
	%\hfill (\ding{55})
\end{problem}



\begin{problem}
	对于初学者来说,代码评审速度可以控制到每小时不超过400行。
    %\hfill (\ding{55})
\end{problem}



\begin{problem}
    ‍“高质量的软件开发是计划出来的”
    %\hfill (\ding{51})
\end{problem}


\subsubsection*{\S\ DevOps概述}
\setcounter{problemname}{0}

\begin{problem}
	下列描述中,不属于典型软件发展三大阶段的是:
	%\uline{C}    
    \vspace{-0.8em}
    \begin{multicols}{4}
        \begin{enumerate}[label=\Alph*.]
            \item 网络化和服务化
            \item 软硬件一体化阶段
            \item 软件作坊
            \item 软件成为独立产品
        \end{enumerate}
    \end{multicols}
    \vspace{-1em}
\end{problem}



\begin{problem}
    ``Measure twice, Cut once" 是哪个阶段的典型开发特征?
    %\uline{D}    
    \vspace{-0.8em}
    \begin{multicols}{2}
        \begin{enumerate}[label=\Alph*.]
            \item 软件作坊阶段
            \item 网络化阶段
            \item 软件成为独立产品阶段
            \item 软硬件一体化阶段
        \end{enumerate}
    \end{multicols}
    \vspace{-1em}
\end{problem}



\begin{problem}
    关于软件过程管理,以下哪一种说法是比较贴切的:
    %\uline{A}    
    % \vspace{-0.8em}
    % \begin{multicols}{2}
        \begin{enumerate}[label=\Alph*.]
            \item 软件过程管理关注的是企业软件过程能力的稳定输出和提升。
            \item 软件过程管理是软件企业发展到较高层次才需要关心的话题。
            \item 软件过程管理主要关注软件成本和质量目标的达成。
            \item 进入互联网时代,软件过程管理是过于老套的话题。
        \end{enumerate}
    % \end{multicols}
    % \vspace{-1em}
\end{problem}



\begin{problem}
    软件开发的本质难题中哪一个与软件发展阶段没有直接关系?
    %\uline{B}    
    \vspace{-0.8em}
    \begin{multicols}{4}
        \begin{enumerate}[label=\Alph*.]
            \item 复杂性
            \item 不可见性
            \item 可变性
            \item 一致性
        \end{enumerate}
    \end{multicols}
    \vspace{-1em}
\end{problem}



\begin{problem}
    ``Code and Fix" 是软件发展哪个阶段的典型开发特征?
    %\uline{C}    
    \vspace{-0.8em}
    \begin{multicols}{4}
        \begin{enumerate}[label=\Alph*.]
            \item 互联网时代
            \item 网络化和服务化
            \item 软硬件一体化
            \item 软件作为独立产品
        \end{enumerate}
    \end{multicols}
    \vspace{-1em}
\end{problem}



\begin{problem}
    以下哪个因素促成了软件成为独立的产品?
    %\uline{D}    
    \vspace{-0.8em}
    \begin{multicols}{2}
        \begin{enumerate}[label=\Alph*.]
            \item 个人电脑的出现
            \item 互联网的出现
            \item 高级程序设计语言的出现
            \item 操作系统的出现
        \end{enumerate}
    \end{multicols}
    \vspace{-1em}
\end{problem}



\begin{problem}
    软件危机和软件工程这两个概念提出时间是?
    %\uline{B}    
    \vspace{-0.8em}
    \begin{multicols}{4}
        \begin{enumerate}[label=\Alph*.]
            \item 上世纪五十年代
            \item 上世纪六十年代
            \item 上世纪八十年代
            \item 上世纪七十年代
        \end{enumerate}
    \end{multicols}
    \vspace{-1em}
\end{problem}



\begin{problem}
    ‍以下描述中,哪几种是网络化和服务化这个阶段的典型软件应用特征?
    %\uline{ABC}    
    % \vspace{-0.8em}
    % \begin{multicols}{2}
        \begin{enumerate}[label=\Alph*.]
            \item 快速演化、需求不确定
            \item 通过SaaS等方式来发布软件系统
            \item 用户数量急剧增加
            \item 通过CD和DVD等方式支持大容量和快速分发软件拷贝
        \end{enumerate}
    % \end{multicols}
    % \vspace{-1em}
\end{problem}


\begin{problem}
    ‌关于形式化方法的描述当中,不正确的有哪些?
    %\uline{BC}    
    % \vspace{-0.8em}
    % \begin{multicols}{2}
        \begin{enumerate}[label=\Alph*.]
            \item 这种方法对开发人员技能有较高的要求
            \item 这种方法的主要目的是解决软件开发的效率问题
            \item 这种方法是网络化和服务化阶段用来应对软件开发本质四大难题而提出来的
            \item 这种方法应用范围有限,例如:不适合跟客户讨论需求。
        \end{enumerate}
    % \end{multicols}
    % \vspace{-1em}
\end{problem}



\begin{problem}
    ‌关于迭代式方法的说法哪些是比较恰当的?
    %\uline{AB}    
    % \vspace{-0.8em}
    % \begin{multicols}{2}
        \begin{enumerate}[label=\Alph*.]
            \item 迭代式方法是指一类具有类似特征的方法
            \item 迭代式方法主要特征在于将软件开发过程视作一个逐步学习和交流的过程
            \item 迭代式方法是上世纪九十年代中后期才出现的一种方法
            \item 迭代式方法主要是为了解决软件开发的质量问题
        \end{enumerate}
    % \end{multicols}
    % \vspace{-1em}
\end{problem}



\begin{problem}
    DevOps方法的出现具有一定的必然性,与以下哪些软件应用特征相匹配?
    %\uline{ABCD}    
    % \vspace{-0.8em}
    % \begin{multicols}{2}
        \begin{enumerate}[label=\Alph*.]
            \item 软件系统部署环境越来越错综复杂
            \item 软件定义世界,软件随处可见
            \item 用户需求多变所带来了软件系统的快速演化的要求
            \item 软件在社会生活当中扮演了越来越关键的角色
        \end{enumerate}
    % \end{multicols}
    % \vspace{-1em}
\end{problem}



\begin{problem}
    DevOps的哪些特点可以有效支撑当前社会对软件系统的期望?
    %\uline{ABCD}    
    % \vspace{-0.8em}
    % \begin{multicols}{2}
        \begin{enumerate}[label=\Alph*.]
            \item 微服务架构设计
            \item 敏捷开发、精益思想以及看板方法,支持快速开发、交付、迭代和演化
            \item 工具链支持高效率的自动化
            \item 虚拟机技术的大量应用
        \end{enumerate}
    % \end{multicols}
    % \vspace{-1em}
\end{problem}



\begin{problem}
    在DevOps化的three ways当中,关注质量问题是第二个阶段才需要考虑的。
    %\hfill ({\ding{55}})
\end{problem}



\begin{problem}
    DevOps中的XaaS特指 SaaS、PaaS以及IaaS这三种。
    %\hfill ({\ding{55}})
\end{problem}



\begin{problem}
    DevOps化的Three ways当中,建立反馈机制是二阶段应该实现的目标。
    %\hfill ({\ding{51}})
\end{problem}


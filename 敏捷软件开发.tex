\subsubsection*{\S 敏捷软件开发}
\setcounter{problemname}{0}

\begin{problem}
	根据敏捷宣言,以下哪项描述了更多的价值?
	\uline{B}    
    % \vspace{-0.8em}
    % \begin{multicols}{2}
        \begin{enumerate}[label=\Alph*.]
            \item 可工作的软件、个体交互、响应变化、相近的文档
            \item 个体和交互、可工作的软件、客户协作、响应变化
            \item 客户协作、遵循计划、可工作的软件、个体交互
            \item 响应变化、个体和交互、流程和工作、客户协作
        \end{enumerate}
    % \end{multicols}
    % \vspace{-1em}
\end{problem}



\begin{problem}
	下列哪一项更好地描述了敏捷宣言?
	\uline{B}    
    \vspace{-0.8em}
    \begin{multicols}{2}
        \begin{enumerate}[label=\Alph*.]
            \item 它包含了许多敏捷团队使用的实践
            \item 它包含了建立敏捷思维方式的价值观
            \item 它定义了构建软件的规则
            \item 它概述了构建软件的最有效方法
        \end{enumerate}
    \end{multicols}
    \vspace{-1em}
\end{problem}




\begin{problem}
	你是一家社交媒体公司的开发人员,正在开发一个项目,项目需要一个为企业客户创建私有网站的新功能。 您需要与公司的网络工程师一起确定部署策略,并提出一组工程师可以用于管理站点的服务和工具。 网络工程师希望在你的网络内部部署所有服务,但您和您的团队成员不同意,并且认为服务应该部署在客户的网络上。 为了达成一个协议,该项目的工作已经停止。 哪种敏捷价值最适合这种情况?
	\uline{A}    
    \vspace{-0.8em}
    \begin{multicols}{2}
        \begin{enumerate}[label=\Alph*.]
            \item 客户合作高于合同谈判
            \item 响应变化高于遵循计划
            \item 个体和互动高于流程和工具
            \item 工作的软件高于详尽的文档
        \end{enumerate}
    \end{multicols}
    \vspace{-1em}
\end{problem}




\begin{problem}
	你是一个软件团队的开发人员。 一个用户向你的团队询问有关构建新功能的信息,并以规范的形式提供了需求。 她非常确定这个功能要如何工作,并承诺不会有任何变化。 哪种敏捷价值最适用于这种情况?
	\uline{B}    
    \vspace{-0.8em}
    \begin{multicols}{2}
        \begin{enumerate}[label=\Alph*.]
            \item 响应变化高于遵循计划
            \item 工作的软件高于详尽的文档
            \item 客户合作高于合同谈判
            \item 个体和互动高于流程和工具
        \end{enumerate}
    \end{multicols}
    \vspace{-1em}
\end{problem}




\begin{problem}
	Sean是一个正在构建财务软件的团队的开发人员。 他的团队被要求开发一个新的交易系统。 他和他的团队召开会议来提出他们正在使用的工作流的图景。 然后,他们将流程放在白板上,流程中的每个步骤都有一列。 经过对团队在白板上的工作项目进行了几周观察,他们注意到这个过程中有几个步骤似乎过载了。对于他们来说,下一步应该做什么?
	\uline{D}    
    % \vspace{-0.8em}
    % \begin{multicols}{2}
        \begin{enumerate}[label=\Alph*.]
            \item 专注于完成看板上的工作
            \item 在较慢的步骤中使用更多的人力
            \item 与团队合作,在工作进展缓慢的阶段更好地完成工作
            \item 对过载步骤中正在进行的工作项目的数量进行限制
        \end{enumerate}
    % \end{multicols}
    % \vspace{-1em}
\end{problem}




\begin{problem}
	‍下列哪一个不是精益原则?
	\uline{A}    
    \vspace{-0.8em}
    \begin{multicols}{4}
        \begin{enumerate}[label=\Alph*.]
            \item 实施反馈循环
            \item 尽可能晚的做决定
            \item 消除浪费
            \item 识别所有的步骤
        \end{enumerate}
    \end{multicols}
    \vspace{-1em}
\end{problem}




\begin{problem}
	下列哪一个更好地描述了如何使用看板?
	\uline{C}    
    % \vspace{-0.8em}
    % \begin{multicols}{2}
        \begin{enumerate}[label=\Alph*.]
            \item 帮助团队自我组织,并了解工作流程中的瓶颈所在
            \item 跟踪缺陷和问题,并创建解决产品问题的最快途径
            \item 观察特征如何流经过程,以便团队可以确定如何限制WIP并通过工作流程中的步骤确定最均匀的工作流程
            \item 跟踪WIP限制和当前任务状态,以便团队知道他们还有多少工作要做
        \end{enumerate}
    % \end{multicols}
    % \vspace{-1em}
\end{problem}




\begin{problem}
	以下不是经常出现在Kanban上记事贴中的内容
	\uline{B}    
    \vspace{-0.8em}
    \begin{multicols}{2}
        \begin{enumerate}[label=\Alph*.]
            \item 完成时间
            \item 团队名词
            \item 谁在处理这个工作项
            \item 工作项描述
        \end{enumerate}
    \end{multicols}
    \vspace{-1em}
\end{problem}




\begin{problem}
	一个公司内,各个团队的Kanban列设置应当一致,便于公司管理。
    \hfill (\ding{55})
\end{problem}




\begin{problem}
	在制品规模越小越好,因为这样可以优化前置时间,并且团队的效率会变高。
    \hfill (\ding{55})
\end{problem}



\begin{problem}
	在DevOps中,可以使用Kanban方法,也可以使用Scrum等其他敏捷方法。
    \hfill (\ding{51})
\end{problem}

